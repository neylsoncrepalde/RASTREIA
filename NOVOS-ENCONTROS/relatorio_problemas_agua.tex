\documentclass[a4paper, 12pt, openright, oneside, english, brazil, article]{abntex2}
\usepackage[brazil]{babel}
\usepackage{graphicx}
\usepackage[utf8]{inputenc}
\usepackage{wrapfig}
\usepackage{lscape}
\usepackage{rotating}
\usepackage{epstopdf}
\usepackage[alf]{abntex2cite}
\usepackage[a4paper, left=3cm, right=2cm, top=3cm, bottom=2cm]{geometry}
\usepackage{indentfirst}
\usepackage{longtable}
\usepackage{amsmath}
\usepackage{verbatim}
\pagestyle{plain}

\titulo{Nota técnica - Análise dos problemas territoriais do acesso à água}
\autor{RASTREIA}


\begin{document}
	\pretextual
	\maketitle
	
	\textual
	
	
	\section{Introdução}
	
	A partir de dados fornecidos pelas instituições \ldots
	
	
	
	\section{Poços fora de operação por município}
	
	% latex table generated in R 3.3.3 by xtable 1.8-2 package
	% Wed Apr 12 10:32:00 2017
	\begin{scriptsize}
	\begin{longtable}{rrl}
		\caption{Poços fora de operação} \\
		\hline
		Frequência & \% Válido & Município \\ 
		\hline
		24.00 & 5.32 & MONTES CLAROS \\ 
		19.00 & 4.21 & SÃO FRANCISCO \\ 
		13.00 & 2.88 & Mato Verde \\ 
		12.00 & 2.66 & Espinosa \\ 
		12.00 & 2.66 & Padre Carvalho \\ 
		12.00 & 2.66 & TAIOBEIRAS \\ 
		10.00 & 2.22 & ARAÇUAÍ \\ 
		10.00 & 2.22 & Catuti \\ 
		10.00 & 2.22 & JAÍBA \\ 
		10.00 & 2.22 & JEQUITINHONHA \\ 
		10.00 & 2.22 & LAGOA DOS PATOS \\ 
		10.00 & 2.22 & MATIAS CARDOSO \\ 
		10.00 & 2.22 & PORTEIRINHA \\ 
		10.00 & 2.22 & RIACHO DOS MACHADOS \\ 
		10.00 & 2.22 & SERRANÓPOLIS DE MINAS \\ 
		8.00 & 1.77 & CARAÍ \\ 
		8.00 & 1.77 & Itamarandiba \\ 
		8.00 & 1.77 & MATO VERDE \\ 
		8.00 & 1.77 & MONTE AZUL \\ 
		8.00 & 1.77 & Monte Azul  \\ 
		8.00 & 1.77 & NINHEIRA \\ 
		6.00 & 1.33 & BERILO \\ 
		6.00 & 1.33 & Chapada do Norte \\ 
		6.00 & 1.33 & Comercinho \\ 
		6.00 & 1.33 & FRUTA DE LEITE \\ 
		6.00 & 1.33 & JENIPAPO DE MINAS \\ 
		6.00 & 1.33 & JOAÍMA \\ 
		6.00 & 1.33 & JORDÂNIA \\ 
		6.00 & 1.33 & MATA VERDE \\ 
		6.00 & 1.33 & MINAS NOVAS \\ 
		6.00 & 1.33 & MIRABELA \\ 
		6.00 & 1.33 & PAI PEDRO \\ 
		6.00 & 1.33 & SALINAS \\ 
		6.00 & 1.33 & SANTA CRUZ DE SALINAS \\ 
		6.00 & 1.33 & SANTA FÉ DE MINAS \\ 
		6.00 & 1.33 & SANTA MARIA DO SUAÇUÍ \\ 
		6.00 & 1.33 & SENADOR MODESTINO GONÇALVES \\ 
		5.00 & 1.11 & Cristália \\ 
		4.00 & 0.89 & Bandeira \\ 
		4.00 & 0.89 & CACHOEIRA DE PAJEÚ \\ 
		4.00 & 0.89 & CORONEL MURTA \\ 
		4.00 & 0.89 & CURRAL DE DENTRO \\ 
		4.00 & 0.89 & Felisburgo \\ 
		4.00 & 0.89 & FRANCISCO BADARÓ \\ 
		4.00 & 0.89 & Francisco Sá \\ 
		4.00 & 0.89 & ICARAÍ DE MINAS \\ 
		4.00 & 0.89 & JACINTO \\ 
		4.00 & 0.89 & JURAMENTO \\ 
		4.00 & 0.89 & LEME DO PRADO \\ 
		4.00 & 0.89 & MEDINA \\ 
		4.00 & 0.89 & MONJOLOS \\ 
		4.00 & 0.89 & PADRE PARAÍSO \\ 
		4.00 & 0.89 & PALMÓPOLIS \\ 
		4.00 & 0.89 & PATIS \\ 
		4.00 & 0.89 & PEDRA AZUL \\ 
		4.00 & 0.89 & PONTO CHIQUE \\ 
		4.00 & 0.89 & PONTO DOS VOLANTES \\ 
		4.00 & 0.89 & RIACHINHO \\ 
		4.00 & 0.89 & RIO DO PRADO \\ 
		4.00 & 0.89 & RUBIM \\ 
		4.00 & 0.89 & SÃO JOÃO DO PARAÍSO \\ 
		4.00 & 0.89 & VARGEM GRANDE DO RIO PARDO \\ 
		4.00 & 0.89 & VÁRZEA DA PALMA \\ 
		3.00 & 0.67 & Janaúba \\ 
		3.00 & 0.67 & Josenópolis \\ 
		2.00 & 0.44 & ARICANDUVA \\ 
		2.00 & 0.44 & BANDEIRA \\ 
		2.00 & 0.44 & Capitão Enéas \\ 
		2.00 & 0.44 & DIVISA ALEGRE \\ 
		2.00 & 0.44 & ITAOBIM \\ 
		2.00 & 0.44 & JAPONVAR \\ 
		2.00 & 0.44 & Rio Pardo de Minas \\ 
		\hline
		\hline
		\label{poc-operacao}
	\end{longtable}
	\end{scriptsize}
	
	
	% Motivos de não operação
	
	\begin{scriptsize}
		% latex table generated in R 3.3.3 by xtable 1.8-2 package
		% Wed Apr 12 11:18:14 2017
		\begin{longtable}{rrr}
			\caption{Motivos de não operação} \\ 
			\hline
			Motivo & Frequência & Percentual \\ 
			\hline
			Sem análise de água & 273.00 & 37.86 \\ 
			Água imprópria para consumo humano & 7.00 & 0.97 \\ 
			Tubulação não entregue & 48.00 & 6.66 \\ 
			Sem energização & 393.00 & 54.51 \\ 
			\hline
			Total & 721.00 & 100.00 \\ 
			\hline
			\hline
			\label{mot-nao-operacao}
		\end{longtable}
	\end{scriptsize}
	
	
	% Municípios por motivo 1
	
	\begin{scriptsize}
		% latex table generated in R 3.3.3 by xtable 1.8-2 package
		% Wed Apr 12 11:30:05 2017
		\begin{longtable}{rrl}
			\caption{Municípios com poços sem análise de água} \\ 
			\hline
			Frequência & Percentual & Município \\ 
			\hline
			24.00 & 8.79 & MONTES CLAROS \\ 
			19.00 & 6.96 & SÃO FRANCISCO \\ 
			16.00 & 5.86 & TAIOBEIRAS \\ 
			12.00 & 4.40 & Padre Carvalho \\ 
			10.00 & 3.66 & ARAÇUAÍ \\ 
			10.00 & 3.66 & LAGOA DOS PATOS \\ 
			10.00 & 3.66 & MATIAS CARDOSO \\ 
			10.00 & 3.66 & PORTEIRINHA \\ 
			10.00 & 3.66 & RIACHO DOS MACHADOS \\ 
			10.00 & 3.66 & SERRANÓPOLIS DE MINAS \\ 
			8.00 & 2.93 & CARAÍ \\ 
			8.00 & 2.93 & MATO VERDE \\ 
			8.00 & 2.93 & MONTE AZUL \\ 
			8.00 & 2.93 & NINHEIRA \\ 
			8.00 & 2.93 & SALINAS \\ 
			6.00 & 2.20 & MATA VERDE \\ 
			6.00 & 2.20 & MINAS NOVAS \\ 
			6.00 & 2.20 & MIRABELA \\ 
			6.00 & 2.20 & PAI PEDRO \\ 
			6.00 & 2.20 & SANTA CRUZ DE SALINAS \\ 
			6.00 & 2.20 & SANTA FÉ DE MINAS \\ 
			6.00 & 2.20 & SENADOR MODESTINO GONÇALVES \\ 
			4.00 & 1.47 & LEME DO PRADO \\ 
			4.00 & 1.47 & MEDINA \\ 
			4.00 & 1.47 & MONJOLOS \\ 
			4.00 & 1.47 & PADRE PARAÍSO \\ 
			4.00 & 1.47 & PALMÓPOLIS \\ 
			4.00 & 1.47 & PATIS \\ 
			4.00 & 1.47 & PEDRA AZUL \\ 
			4.00 & 1.47 & PONTO CHIQUE \\ 
			4.00 & 1.47 & PONTO DOS VOLANTES \\ 
			4.00 & 1.47 & RIACHINHO \\ 
			4.00 & 1.47 & RIO DO PRADO \\ 
			4.00 & 1.47 & RUBIM \\ 
			4.00 & 1.47 & SÃO JOÃO DO PARAÍSO \\ 
			4.00 & 1.47 & VARGEM GRANDE DO RIO PARDO \\ 
			4.00 & 1.47 & VÁRZEA DA PALMA \\ 
			\hline
			\hline
			\label{motivo1}
		\end{longtable}
	\end{scriptsize}
	
	
	% Motivo água imprópria para o consumo
	
	\begin{scriptsize}
		% latex table generated in R 3.3.3 by xtable 1.8-2 package
		% Wed Apr 12 11:35:42 2017
		\begin{longtable}{rrl}
			\caption{Municípios com poços com água imprópria para o consumo} \\ 
			\hline
			Frequência & Percentual & Município \\ 
			\hline
			5.00 & 71.43 & Cristália \\ 
			2.00 & 28.57 & ARICANDUVA \\ 
			\hline
			\hline
			\label{motivo2}
		\end{longtable}
	\end{scriptsize}


	% Motivo Tubulação não entregue
	
	\begin{scriptsize}
		% latex table generated in R 3.3.3 by xtable 1.8-2 package
		% Wed Apr 12 11:37:38 2017
		\begin{longtable}{rrl}
			\caption{Municípios com poços com tubulação não entregue} \\ 
			\hline
			Frequência & Percentual & Município \\ 
			\hline
			10.00 & 20.83 & RIACHO DOS MACHADOS \\ 
			10.00 & 20.83 & SERRANÓPOLIS DE MINAS \\ 
			6.00 & 12.50 & SANTA FÉ DE MINAS \\ 
			6.00 & 12.50 & SANTA MARIA DO SUAÇUÍ \\ 
			4.00 & 8.33 & JURAMENTO \\ 
			4.00 & 8.33 & RIO DO PRADO \\ 
			4.00 & 8.33 & SALINAS \\ 
			4.00 & 8.33 & VÁRZEA DA PALMA \\ 
			\hline
			\hline
			\label{motivo3}
		\end{longtable}
	\end{scriptsize}

	% Motivo Sem energização
	
	\begin{scriptsize}
		% latex table generated in R 3.3.3 by xtable 1.8-2 package
		% Wed Apr 12 11:39:02 2017
		\begin{longtable}{rrl}
			\caption{Municípios com poços sem energização} \\ 
			\hline
			Frequência & Percentual & Município \\ 
			\hline
			24.00 & 6.11 & MONTES CLAROS \\ 
			17.00 & 4.33 & SÃO FRANCISCO \\ 
			12.00 & 3.05 & Padre Carvalho \\ 
			12.00 & 3.05 & TAIOBEIRAS \\ 
			10.00 & 2.54 & Catuti \\ 
			10.00 & 2.54 & JAÍBA \\ 
			10.00 & 2.54 & LAGOA DOS PATOS \\ 
			10.00 & 2.54 & MATIAS CARDOSO \\ 
			10.00 & 2.54 & PORTEIRINHA \\ 
			10.00 & 2.54 & RIACHO DOS MACHADOS \\ 
			10.00 & 2.54 & SERRANÓPOLIS DE MINAS \\ 
			8.00 & 2.04 & ARAÇUAÍ \\ 
			8.00 & 2.04 & CARAÍ \\ 
			8.00 & 2.04 & MATO VERDE \\ 
			8.00 & 2.04 & MONTE AZUL \\ 
			8.00 & 2.04 & SALINAS \\ 
			6.00 & 1.53 & BERILO \\ 
			6.00 & 1.53 & Chapada do Norte \\ 
			6.00 & 1.53 & Comercinho \\ 
			6.00 & 1.53 & FRUTA DE LEITE \\ 
			6.00 & 1.53 & Itamarandiba \\ 
			6.00 & 1.53 & JENIPAPO DE MINAS \\ 
			6.00 & 1.53 & JEQUITINHONHA \\ 
			6.00 & 1.53 & JOAÍMA \\ 
			6.00 & 1.53 & MATA VERDE \\ 
			6.00 & 1.53 & MINAS NOVAS \\ 
			6.00 & 1.53 & MIRABELA \\ 
			6.00 & 1.53 & PAI PEDRO \\ 
			6.00 & 1.53 & SANTA CRUZ DE SALINAS \\ 
			6.00 & 1.53 & SANTA FÉ DE MINAS \\ 
			6.00 & 1.53 & SANTA MARIA DO SUAÇUÍ \\ 
			6.00 & 1.53 & SENADOR MODESTINO GONÇALVES \\ 
			5.00 & 1.27 & Cristália \\ 
			4.00 & 1.02 & CACHOEIRA DE PAJEÚ \\ 
			4.00 & 1.02 & CORONEL MURTA \\ 
			4.00 & 1.02 & Felisburgo \\ 
			4.00 & 1.02 & FRANCISCO BADARÓ \\ 
			4.00 & 1.02 & Francisco Sá \\ 
			4.00 & 1.02 & ICARAÍ DE MINAS \\ 
			4.00 & 1.02 & JACINTO \\ 
			4.00 & 1.02 & JORDÂNIA \\ 
			4.00 & 1.02 & JURAMENTO \\ 
			4.00 & 1.02 & LEME DO PRADO \\ 
			4.00 & 1.02 & MEDINA \\ 
			4.00 & 1.02 & MONJOLOS \\ 
			4.00 & 1.02 & PADRE PARAÍSO \\ 
			4.00 & 1.02 & PALMÓPOLIS \\ 
			4.00 & 1.02 & PATIS \\ 
			4.00 & 1.02 & PEDRA AZUL \\ 
			4.00 & 1.02 & PONTO CHIQUE \\ 
			4.00 & 1.02 & PONTO DOS VOLANTES \\ 
			4.00 & 1.02 & RIACHINHO \\ 
			4.00 & 1.02 & RIO DO PRADO \\ 
			4.00 & 1.02 & RUBIM \\ 
			4.00 & 1.02 & SÃO JOÃO DO PARAÍSO \\ 
			4.00 & 1.02 & VARGEM GRANDE DO RIO PARDO \\ 
			4.00 & 1.02 & VÁRZEA DA PALMA \\ 
			3.00 & 0.76 & Josenópolis \\ 
			2.00 & 0.51 & ARICANDUVA \\ 
			2.00 & 0.51 & Bandeira \\ 
			2.00 & 0.51 & BANDEIRA \\ 
			2.00 & 0.51 & Capitão Enéas \\ 
			2.00 & 0.51 & CURRAL DE DENTRO \\ 
			2.00 & 0.51 & DIVISA ALEGRE \\ 
			2.00 & 0.51 & ITAOBIM \\ 
			2.00 & 0.51 & JAPONVAR \\ 
			2.00 & 0.51 & NINHEIRA \\ 
			\hline
			\hline
			\label{motivo4}
		\end{longtable}
	\end{scriptsize}

	
	\section{Problemas fundiários}
	Apenas 8\% do universo investigado apresenta dados de problemas fundiários. A grande quantidade de não respostas (\textit{missing values}) torna inviável a análise desta variável através dos dados disponibilizados.
	
	
	\section{Licença ambiental}
	Verificamos primeiro a distribuição de frequências das categorias da variável - ver tabela \ref{outorga}.
	
	\begin{scriptsize}
		% latex table generated in R 3.3.3 by xtable 1.8-2 package
		% Wed Apr 12 11:49:33 2017
		\begin{longtable}{rrrr}
			\caption{Licença ambiental - possui outorga?} \\ 
			\hline
			& Frequência & Percentual & \% Válido \\ 
			\hline
			Não & 445.00 & 57.72 & 76.99 \\ 
			Dispensa de Licenciamento & 96.00 & 12.45 & 16.61 \\ 
			Outorgado pelo IGAM & 36.00 & 4.67 & 6.23 \\ 
			Pendente & 1.00 & 0.13 & 0.17 \\ 
			NA's & 193.00 & 25.03 &  \\ 
			\hline
			Total & 771.00 & 100.00 & 100.00 \\ 
			\hline
			\hline
			\label{outorga}
		\end{longtable}
	\end{scriptsize}

	
	% Municípios sem outorga
	\begin{scriptsize}
		% latex table generated in R 3.3.3 by xtable 1.8-2 package
		% Wed Apr 12 12:00:14 2017
		\begin{longtable}{rrl}
			\caption{Municípios com poços sem outorga} \\ 
			\hline
			Frequência & Percentual & Município \\ 
			\hline
			24.00 & 5.39 & MONTES CLAROS \\ 
			19.00 & 4.27 & SÃO FRANCISCO \\ 
			16.00 & 3.60 & TAIOBEIRAS \\ 
			12.00 & 2.70 & Padre Carvalho \\ 
			10.00 & 2.25 & ARAÇUAÍ \\ 
			10.00 & 2.25 & Catuti \\ 
			10.00 & 2.25 & JAÍBA \\ 
			10.00 & 2.25 & JEQUITINHONHA \\ 
			10.00 & 2.25 & LAGOA DOS PATOS \\ 
			10.00 & 2.25 & MATIAS CARDOSO \\ 
			10.00 & 2.25 & MATO VERDE \\ 
			10.00 & 2.25 & PORTEIRINHA \\ 
			10.00 & 2.25 & RIACHO DOS MACHADOS \\ 
			10.00 & 2.25 & SALINAS \\ 
			10.00 & 2.25 & SERRANÓPOLIS DE MINAS \\ 
			8.00 & 1.80 & CARAÍ \\ 
			8.00 & 1.80 & Itamarandiba \\ 
			8.00 & 1.80 & MONTE AZUL \\ 
			8.00 & 1.80 & NINHEIRA \\ 
			6.00 & 1.35 & BERILO \\ 
			6.00 & 1.35 & Chapada do Norte \\ 
			6.00 & 1.35 & Comercinho \\ 
			6.00 & 1.35 & Francisco Sá \\ 
			6.00 & 1.35 & FRUTA DE LEITE \\ 
			6.00 & 1.35 & JAPONVAR \\ 
			6.00 & 1.35 & JENIPAPO DE MINAS \\ 
			6.00 & 1.35 & JOAÍMA \\ 
			6.00 & 1.35 & JORDÂNIA \\ 
			6.00 & 1.35 & LEME DO PRADO \\ 
			6.00 & 1.35 & MATA VERDE \\ 
			6.00 & 1.35 & MEDINA \\ 
			6.00 & 1.35 & MINAS NOVAS \\ 
			6.00 & 1.35 & MIRABELA \\ 
			6.00 & 1.35 & PAI PEDRO \\ 
			6.00 & 1.35 & PEDRA AZUL \\ 
			6.00 & 1.35 & SANTA CRUZ DE SALINAS \\ 
			6.00 & 1.35 & SANTA FÉ DE MINAS \\ 
			6.00 & 1.35 & SANTA MARIA DO SUAÇUÍ \\ 
			6.00 & 1.35 & SENADOR MODESTINO GONÇALVES \\ 
			6.00 & 1.35 & VIRGEM DA LAPA \\ 
			5.00 & 1.12 & Cristália \\ 
			4.00 & 0.90 & Bandeira \\ 
			4.00 & 0.90 & CACHOEIRA DE PAJEÚ \\ 
			4.00 & 0.90 & CARBONITA \\ 
			4.00 & 0.90 & CORONEL MURTA \\ 
			4.00 & 0.90 & CURRAL DE DENTRO \\ 
			4.00 & 0.90 & Felisburgo \\ 
			4.00 & 0.90 & FRANCISCO BADARÓ \\ 
			4.00 & 0.90 & ICARAÍ DE MINAS \\ 
			4.00 & 0.90 & JACINTO \\ 
			4.00 & 0.90 & JURAMENTO \\ 
			4.00 & 0.90 & MONJOLOS \\ 
			4.00 & 0.90 & PADRE PARAÍSO \\ 
			4.00 & 0.90 & PALMÓPOLIS \\ 
			4.00 & 0.90 & PATIS \\ 
			4.00 & 0.90 & PONTO CHIQUE \\ 
			4.00 & 0.90 & PONTO DOS VOLANTES \\ 
			4.00 & 0.90 & RIACHINHO \\ 
			4.00 & 0.90 & RIO DO PRADO \\ 
			4.00 & 0.90 & RUBIM \\ 
			4.00 & 0.90 & SÃO JOÃO DO PARAÍSO \\ 
			4.00 & 0.90 & VARGEM GRANDE DO RIO PARDO \\ 
			4.00 & 0.90 & VÁRZEA DA PALMA \\ 
			3.00 & 0.67 & Josenópolis \\ 
			2.00 & 0.45 & ARICANDUVA \\ 
			2.00 & 0.45 & BANDEIRA \\ 
			2.00 & 0.45 & Capitão Enéas \\ 
			2.00 & 0.45 & DIVISA ALEGRE \\ 
			2.00 & 0.45 & ITAOBIM \\ 
			\hline
			\hline
			\label{semoutorga}
		\end{longtable}
	\end{scriptsize}	
	
	
	
	
	
	
	
\end{document}
