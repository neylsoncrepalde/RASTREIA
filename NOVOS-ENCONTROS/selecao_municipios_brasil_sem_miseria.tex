\documentclass[a4paper, 12pt, openright, oneside, english, brazil, article]{abntex2}
\usepackage[brazil]{babel}
\usepackage{graphicx}
\usepackage[utf8]{inputenc}
\usepackage{wrapfig}
\usepackage{lscape}
\usepackage{rotating}
\usepackage{epstopdf}
\usepackage[alf]{abntex2cite}
\usepackage[a4paper, left=3cm, right=2cm, top=3cm, bottom=2cm]{geometry}
\usepackage{indentfirst}
\usepackage{longtable}
\usepackage{amsmath}
\usepackage{verbatim}
\pagestyle{plain}

\titulo{Nota técnica - seleção de municípios para ações Brasil sem Miséria e Kits Alimento e Trabalho}
\autor{RASTREIA}



\begin{document}
	\pretextual
	\maketitle
	
	\textual
	

	\section{Introdução}	
		Os resultados dos modelos estatísticos estimados encontram-se na tabela \ref{table:reg}. O Índice de Correlação Intraclasse (ICC) indica a porcentagem da variância dos dados explicados pelos grupos de 2º nível, neste caso, os municípios\footnote{Ver CREPALDE, Neylson João Batista Filho; SILVEIRA, Leonardo Souza. Desempenho universitário no Brasil: estudo sobre desigualdade educacional com dados do Enade 2014. \textbf{Revista Brasileira de Sociologia-RBS}, v. 4, n. 7, 2016.}. O ICC pode ser definido por
		
		$$ICC = \frac{Var(U_{0j})}{Var(U_{0j} + Var(e_{ij}} = \frac{\tau_{00}}{\tau_{00} + \sigma^2} $$
		
		O ICC calculado indica que os municípios correspondem a 45,3\% da variância dos dados, ou seja, assumem um peso considerável na estimação de condições de pobreza.
		
		\begin{table}[!h]
			\ibgetab{
			\centering
			\caption{Modelos estatísticos}
			\label{table:reg}
		}
			{\begin{tabular}{l c c}
				\hline
				& Logístico & Logístico Multinível \\
				& Razão de chance (\%)  & Razão de chance (\%) \\
				\hline
				(Intercepto)                                                 & $9.06^{***}$  & $90.51^{***}$  \\
				& $(0.01)$      & $(0.04)$      \\
				Local do Domicílio - Rurais                                & $35.67^{***}$  & $23.11^{***}$  \\
				& $(0.01)$      & $(0.01)$      \\
				Qtd de cômodos dormitórios                               & $-24.15^{***}$ & $-32.56^{***}$ \\
				& $(0.00)$      & $(0.00)$      \\
				Tem água canalizada - Não                               & $11.14^{***}$  & $7.08^{***}$  \\
				& $(0.01)$      & $(0.01)$      \\
				Forma de abastecimento de água &   &   \\
				Poço ou nascente               & $31.17^{***}$  & $17.47^{***}$  \\
				& $(0.01)$      & $(0.01)$      \\
				Cisterna                       & $29.08^{***}$  & $29.39^{***}$  \\
				& $(0.02)$      & $(0.02)$      \\
				Outra forma                    & $21.13^{***}$  & $15.85^{***}$  \\
				& $(0.02)$      & $(0.02)$      \\
				\textbf{Existência de banheiro} - Não                 & $66.59^{***}$  & $46.43^{***}$  \\
				& $(0.01)$      & $(0.01)$      \\
				\textbf{Tipo de iluminação}   &    &    \\
				Elétrica com medidor comunitário & $11.18^{***}$  & $44.97^{***}$  \\
				& $(0.01)$      & $(0.01)$      \\
				Elétrica sem medidor             & $84.78^{***}$  & $53.19^{***}$  \\
				& $(0.02)$      & $(0.02)$      \\
				Óleo, querosene ou gás           & $77.33^{***}$  & $68.46^{***}$  \\
				& $(0.03)$      & $(0.03)$      \\
				Vela                             & $144.19^{***}$  & $120.53^{***}$  \\
				& $(0.04)$      & $(0.04)$      \\
				Outra forma                      & $67.17^{***}$  & $39.07^{***}$  \\
				& $(0.03)$      & $(0.02)$      \\
				\hline
				AIC                                                         & 846088.62     & 771288.68     \\
				BIC                                                         & 846236.41     & 771447.83     \\
				Log Likelihood                                              & -423031.31    & -385630.34    \\
				Deviance                                                    & 846062.62     &               \\
				Num. obs.                                                   & 639373        & 639373        \\
				Num. groups: nome\_munic                                    &               & 229           \\
				Var: nome\_munic (Intercept)                                &               & 0.47          \\
				\hline
				\multicolumn{3}{l}{\scriptsize{$^{***}p<0.001$, $^{**}p<0.01$, $^*p<0.05$}}
			\end{tabular}
			}
			{\fonte{Elaboração própria a partir de dados do CADUNICO.}}
		\end{table}
	

		Os resultados do modelo estão apresentados em Razão de Chance ($\exp(\beta - 1) * 100$) e podem ser interpretados como porcentagens de razão de chance em relação ao sucesso de ``ter renda per capita mensal de até R\$85,00''. Os dados indicam que os preditores com maior peso na variável resposta são \textbf{Tipo de Iluminação} e \textbf{Existência de Banheiro}. Consideramos, portanto, que essas variáveis devem ser incorporadas ao filtro de seleção de domicílios para recebimento dos kits.
	
	
\end{document}