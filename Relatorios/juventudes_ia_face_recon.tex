\documentclass[a4paper, 12pt, openright, oneside, english, brazil, article]{abntex2}
\usepackage[brazil]{babel}
\usepackage{graphicx}
\usepackage[utf8]{inputenc}
\usepackage{graphicx}
\usepackage{wrapfig}
\usepackage{lscape}
\usepackage{rotating}
\usepackage{epstopdf}
\usepackage[alf]{abntex2cite}
\usepackage[a4paper, left=3cm, right=2cm, top=3cm, bottom=2cm]{geometry}
\usepackage{indentfirst}
\usepackage{longtable}
\usepackage{amsmath}
\usepackage{algorithm}
\floatname{algorithm}{Código}
\renewcommand{\listalgorithmname}{Lista de Códigos}
\usepackage{algpseudocode} %para escrever pseudo-algoritmos
\usepackage{TikZ}
\usepackage{listings} %para escrever códigos

\pagestyle{plain}

\titulo{\textbf{O uso de inteligência artificial no acompanhamento do programa Juventudes}}
\autor{Neylson Crepalde, PhD candidate
		\par
		Wesley Matheus, PhD candidate
		\par
		Pedro Pinto}
\data{Fevereiro, 2017}
\instituicao{RASTREIA
	\par
	SEDESE -- MG}
\local{Belo Horizonte}
\tipotrabalho{Relatório técnico}

\begin{document}
	
	\thispagestyle{empty}
	\begin{tikzpicture}[remember picture, overlay]
	\node[inner sep=0pt] at (current page.center) {%
		\includegraphics[width=\paperwidth,height=\paperheight]{Capa_rastreia.png}%
	};
	\end{tikzpicture}
	\cleardoublepage{}
	

	\newpage
	
	\imprimirfolhaderosto
	
	%\listofalgorithms
	%\listoffigures
	%\listoftables	
	%\newpage
	%\tableofcontents
	\textual
	
	
	\section*{Introdução}
	
	Muito embora estejamos vivenciando um período de grandes inovações e avanços tecnológicos, informacionais e computacionais como nunca antes vistos, o acesso a esse conhecimento bem como seu uso na máquina estatal é ainda sobremameira restrito. A despeito disso, a Diretoria de Monitoramento e Avaliação, doravante \textbf{RASTREIA}, busca sanar essa lacuna aproximando a gestão pública desses avanços visando sua aplicação no monitoramento contínuo das ações da Secretaria do Estado de Trabalho e Desenvolvimento Social e do Governo do Estado de Minas Gerais.
	
	\section*{O sistema \textit{Acompanha}}
	
	O sistema \textit{Acompanha} foi desenvolvido na SEDESE com o intuito de acompanhar e oferecer um suporte aos alunos participantes das oficinas oferecidas dentro do programa \textit{Juventudes}. O sistema recebe informações sobre os alunos, sobre os professores e sobre a turma. Um dos objetivos do sistema é verificar a quantidade de alunos atendidos pelo programa em cada regional. A fiscalização da veracidade das informações é demasiado custosa dado o grande número de alunos atendidos pelo programa e sua distribuição em 4 municípios (9 regiões) do estado mineiro. Como uma solução para esse problema, as instituições serão solicitadas a enviar algumas fotos da turma ao longo das atividades formativas as quais serão utilizadas para verificação do número de alunos através de contagem de rostos automatizada (ver Figura \ref{face_class}). Explicitaremos abaixo, mais detidamente, de que modo a contagem de rostos é realizada.
	
	\begin{figure}[!h]
		\centering
		\caption{Aplicação de teste com OpenCV}
		\label{face_class}
		\includegraphics[scale=.45]{face_reco1.png}
	\end{figure}
	
	\section*{\textit{Computer Vision}}
	
	O OpenCV (\textit{Open Source Computer Vision}) é um módulo computacional amplamente utilizado para processamento de imagens que começou a ser desenvolvido pela \textbf{Intel} no final da década de 1990. O módulo trabalha com 3 algoritmos de classificação, a saber, \textit{Eigenfaces}, \textit{Fisherfaces} e \textit{Local Binary Patterns Histograms}. Foi demonstrado\footnote{De acordo com documentação oficial do módulo. Disponível em \url{http://docs.opencv.org/3.1.0/da/d60/tutorial_face_main.html}, acesso em 08 fev 2017.} que os algoritmos disponíveis proveem resultados bastante parecidos. Explicitaremos, em linhas gerais, a lógica por trás do algoritmo.
	
	\subsection*{O método}
	
	O método \textit{Eigenfaces} (padrão do módulo) usa da estrutura geométrica da face para efetuar o reconhecimento: ``Uma imagem facial é um ponto no espaço de uma imagem de alta dimensionalidade e uma representação de menor dimensão é encontrada onde a classificação torna-se fácil\footnote{A facial image is a point from a high-dimensional image space and a lower-dimensional representation is found, where classification becomes easy.}''. Esse subespaço de menor dimensão é encontrado com Análise de Componentes Principais, uma técnica matemática de análise multivariada. Uma descrição formal do método \textit{Eigenfaces} é apresentada na nota metodológica em anexo.
	
	\section*{Análise dos dados}
	
	No contexto deste programa, serão utilizados quatro classificadores faciais já treinados e implementados no OpenCV. Assumindo que técnicas de inteligência artificial geram variabilidade nos resultados e, ao mesmo tempo, buscando a maior acurácia possível no processo, os resultados gerados por cada classificador serão analisados a partir de três estatísticas descritivas, a saber, a média, a moda e a variância. Esses indicadores possibilitam saber quando os classificadores estão em alta concordância ou quando a variabilidade dos dados fica demasiado alta, comprometendo a contagem automatizada de rostos. A partir disso, o RASTREIA pode identificar manualmente os casos com alta variabilidade e monitorar por amostra os resultados da contagem para as demais regiões.
	
	Essa inovação possibilita o uso inteligente dos dados, dos recursos à disposição, de tecnologia de ponta via utilização de ferramentas de código aberto bem como transparência nos processos.
	
	
	\newpage
	\postextual
	\anexos
	\section*{Anexo - Nota Metodológica}
	Descrevemos abaixo o procedimento do algoritmo \textit{Eigenfaces} para classificação facial:
	
	Seja $X = \{x_1, x_2, \ldots, x_n\}$ um vetor aleatório com observações $x_i \in R^d$.
	
	\begin{enumerate}
		\item Calcula-se a média $\mu$
		$$\mu = \frac{1}{n} \sum_{i=1}^{n}x_i$$
		
		\item Calcula-se a matrix de covariância $S$
		$$S = \frac{1}{n} \sum_{i=1}^{n}(x_i - \mu)(x_i - \mu)^T$$
		
		\item Calcula-se os autovalores $\lambda_i$ e autovetores $v_i$ de $S$
		$$Sv_i = \lambda_iv_i, i=1,2,\ldots,n$$
		
		\item Ordena-se os autovetores pelos valores descendentes de seu autovalor. Os $k$ componentes principais são os autovetores correspondentes aos $k$ maiores autovalores.		
	\end{enumerate}

	Os $k$ componentes principais do vetor $x$ observado são, portanto, dados por:
	$$y = W^T(x-\mu)$$
	onde $W = (v_1, v_2, \ldots, v_k)$. 
	
	A reconstrução da base PCA é dada por:
	$$x = Wy + \mu$$

	\textit{Eigenfaces} então efetua o reconhecimento facial:
	\begin{itemize}
		\item Projetando todas as amostras de treinamento no espaço do PCA.
		\item Projetando a imagem analisada no espaço do PCA.
		\item Encontrando o vizinho mais próximo (\textit{nearest neighbor}) entre as imagens de treino projetadas e a imagem investigada projetada.
	\end{itemize}

	
	
	
	
	\newpage
	
	\thispagestyle{empty}
	\begin{tikzpicture}[remember picture, overlay]
	\node[inner sep=0pt] at (current page.center) {%
		\includegraphics[width=\paperwidth,height=\paperheight]{Contracapa_rastreia.png}%
	};
	\end{tikzpicture}
	\cleardoublepage{}
	
\end{document}
