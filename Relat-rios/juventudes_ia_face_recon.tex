\documentclass[a4paper, 12pt, openright, oneside, german, french, english, brazil, article]{abntex2}
\usepackage[brazil]{babel}
\usepackage{graphicx}
\usepackage[utf8]{inputenc}
\usepackage{graphicx}
\usepackage{wrapfig}
\usepackage{lscape}
\usepackage{rotating}
\usepackage{epstopdf}
\usepackage[alf]{abntex2cite}
\usepackage[a4paper, left=3cm, right=2cm, top=3cm, bottom=2cm]{geometry}
\usepackage{indentfirst}
\usepackage{longtable}
\usepackage{amsmath}
\usepackage{algorithm}
\floatname{algorithm}{Código}
\renewcommand{\listalgorithmname}{Lista de Códigos}
\usepackage{algpseudocode} %para escrever pseudo-algoritmos
\usepackage{listings} %para escrever códigos
\pagestyle{plain}

\titulo{\textbf{O uso de inteligência artificial no acompanhamento do programa Juventudes}}
\autor{Neylson J. B. F. Crepalde, PhD candidate}
\data{Fevereiro, 2017}
\instituicao{RASTREIA
	\par
	SEDESE -- MG}
\local{Belo Horizonte}
\tipotrabalho{Relatório técnico}

\begin{document}
	
	\imprimirfolhaderosto
	
	%\listofalgorithms
	%\listoffigures
	%\listoftables	
	%\newpage
	%\tableofcontents
	\textual
	
	
	\section*{Introdução}
	
	Muito embora estejamos vivenciando um período de grandes inovações e avanços tecnológicos, informacionais e computacionais como nunca antes vistos, o acesso a esse conhecimento bem como seu uso na máquina estatal é ainda sobremameira restrito. A despeito disso, a Diretoria de Monitaramento e Avaliação em seu projeto intitulado \textbf{RASTREIA} busca sanar esta lacuna aproximando a gestão pública desses avanços visando sua aplicação no monitoramento contínuo das ações da Secretaria do Estado de Trabalho e Desenvolvimento Social e do Estado de Minas Gerais.
	
	\section*{O sistema \textit{Acompanha} -- REVISAR}
	
	O sistema \textit{Acompanha} foi desenvolvido na SEDESE com o intuito de acompanhar e oferecer um suporte aos alunos participantes das oficinas oferecidas dentro do programa \textit{Juventudes}. O sistema recebe informações sobre os alunos, sobre os professores e sobre a turma. Um dos objetivos do sistema é verificar a quantidade de alunos atendidos pelo programa em cada regional. A fiscalização da veracidade das informações alimentadas no sistema pelos gestores das regionais é demasiado custosa dado o grande número de alunos atendidos pelo programa e sua distribuição em X localidades do estado mineiro. Como uma solução para esse problema, os gestores das regionais serão incentivados a enviar algumas fotos da turma ao longo das atividades formativas as quais serão utilizadas para verificação do número de alunos através de contagem de rostos automatizada. Explicitamos mais detidamente de que modo a contagem de rostos em fotos é realizada.
	
	\section{\textit{Computer Vision}}
	
	O OpenCV (CONTINUA)\ldots
	
	
	
\end{document}
